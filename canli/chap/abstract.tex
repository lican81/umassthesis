\begin{abstract}

In the past decades, the computing capability has shown an exponential growth trend, which is observed as Moore’s law. However, this growth speed is slowing down	in recent years mostly because the down-scaled size of transistors is approaching their physical limit. On the other hand, recent advances in software, especially in big data analysis and artificial intelligence, call for a break-through in computing hardware. 
Memristor, or resistive switching device, is believed to be a potential building block of the future generation of integrated circuits. The underlying mechanism of this device is different from that of complementary metal-oxide-semiconductor (CMOS) transistors, which provides better scaling potential than that of CMOS transistors to make the Moore’s law last longer. More importantly, the resistance of the of the two-terminal device, depends on the history of the applied voltages, and therefore the computing based on this device takes place in the exact location where information is stored. As the result, a revolutionary computing machine based on this device circumvent the need of data transfer between computing unit and memory unit in a conventional von-Neumann machine. 

There are still challenges to build memristor based machine to solve real-world applications. As an emerging device, despite promising properties demonstrated in single device level, it is still not mature enough to make integrated circuit chip with decent array performance, mostly due to large spatial and temporal variation. 
The problems may be solved in the future by continuous device and material engineering, they can also be remedied with the help of the mature CMOS technology, while maintaining most advantages that memristor provides. In this dissertation, we present our experimental work to integrate memristors with CMOS circuitry and demonstrate real-world applications. Firstly, we explore the possibility to use CMOS foundry compatible material (e.g. silicon oxide and hafnium oxide) to make working memristors. After that we study the advanced memristor fabrication technology including three-dimensional stacking and foundry compatible integration with transistors. The maturity of the integrated memristor chip is then demonstrated by real-world applications, which includes an ex-situ method to precisely calculate matrix multiplication for signal and image processing and an in-situ method to compensate defects and variability for training of neural networks. The calculation based on this system suggests that, with integrated peripherals which will be available in the near future, the memristor based system gains significant advantages over the conventional digital CMOS approaches in both speed and energy efficiency.

\end{abstract}